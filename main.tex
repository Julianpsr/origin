\documentclass{article}
\usepackage[utf8]{inputenc}
\usepackage[spanish]{babel}
\usepackage{listings}
\usepackage{graphicx}
\graphicspath{ {images/} }
\usepackage{cite}

\begin{document}

\begin{titlepage}
    \begin{center}
        \vspace*{1cm}
            
        \Huge
        \textbf{Calistenia}
            
        \vspace{0.5cm}
        \LARGE
            
        \vspace{1.5cm}
            
        \textbf{Julián David Taborda Ramírez}
            
        \vfill
            
        \vspace{0.8cm}
            
        \Large
        Despartamento de Ingeniería Electrónica y Telecomunicaciones\\
        Universidad de Antioquia\\
        Medellín\\
        Marzo de 2021 
            
    \end{center}
\end{titlepage}

\tableofcontents
\newpage
\section{Sección introductoria}\label{intro}
Vamos a analizar el desarrollo del siguiente desafío, que consta en llevar dos objetos de una posición X a una posición Y. Teniendo en cuenta algunas consideraciones especiales para solucionar el ejercicio correctamente. Además la resolución contiene un paso a paso detallado de cada uno de los estados que se van a presentar.
\section{Sección de contenido} \label{contenido}
Descripción del desarrollo del desafío:

1. Observar el estado inicial del sistema, siguiendo las condiciones iniciales.

2. Tomar la hoja que se encuentra sobre las tarjetas, teniendo en cuenta que debe ser con una sola mano. 

3. Levantar la hoja, luego posicionarla sobre una superficie liza de manera horizontal, evitando que la hoja interrumpa la trayectoria que debe seguir las tarjetas con su mano. 

4. Luego de tener la hoja ubicada en el lugar idóneo, tomar las tarjetas de la siguiente manera:

    •	Los dedos pulgar y medio, deben ir al costado de las tarjetas. Siendo el borde más largo el requerido. Ahora ubique los dedos respectivamente en cada borde.  

    Defínase: El borde más largo como aquel que su longitud es el doble que el borde más corto, para este caso específico.  

5. Levantar las tarjetas. A continuación, debes posicionar las tarjetas en sentido vertical.  

6. Ubicar el dedo índice sobre el borde más corto y el superior de las tarjetas en el estado vertical, ubicando el dedo en el centro de este borde. 

    Defínase: El centro del borde más corto como la parte equidistante de los dos extremos de la tarjeta.   

7.  Superponer las tarjetas sobre la hoja. Ahora ejerciendo presión con el dedo índice comenzamos a generar un pequeño espacio en la zona que este en contacto con la hoja. Recuerde el dedo índice aún debe estar sobre las tarjetas. 
    (La tarjeta más interna debe estar generando un rozamiento a medida que se separa de la otra)

8. Ahora cuidadosamente alejamos los dedos pulgar y medio de la tarjeta más externa a la mano. Ahora nótese una sensación de equilibrio a medida que el espacio entre las tarjetas aumenta. 

9. Evaluamos la relación equilibrio-separación, haciendo pequeñas pruebas levantando el índice. Cuando la separación entre las tarjetas sea equilibrada, pause el movimiento y retire su mano lentamente. 

10. Observe el estado final del sistema.  

\section{Conclusiones} 
Gracias a esta actividad se puede comprobar como las instrucciones explicitas, juegan un papel muy importante a la hora de imponer orden y exactitud a muchas de nuestras acciones. Ahora llevando este caso a un contexto informático concluimos que la lógica es el pilar fundamental o la herramienta necesaria, para concebir muchos de los conceptos informáticos que serán de ayuda para construir sentencias lógicas y posteriormente programas informáticos.


\end{document}
